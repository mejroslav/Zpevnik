\begin{song}
{
title = {Nagasaki, Hirošima},
interpret = {Karel Plíhal}
}

\begin{verse}
^{C}Tramvají ^{G}dvojkou ^{F}jezdíval jsem ^{G}do \chord{C}Židenic,  \chord{G F G} \\
^{C}z~tak velký ^{G}lásky ^{F}většinou ^{G}nezbyde ^{Am}nic, \chord{Em} \\
^{F}z~takový ^{C}lásky ^{F}jsou kruhy ^{C}pod ^{G}očima \\
a dvě ^{C}spálený ^{G}srdce – ^{F}Nagasaki, ^{G}Hirošima. \chord{C G F G}
\end{verse}

\begin{verse}
Jsou jistý věci, co bych tesal do kamene, \\
tam, kde je láska, tam je všechno dovolené, \\
a tam, kde není, tam mě to nezajímá, \\
jó, dvě spálený srdce – Nagasaki, Hirošima.
\end{verse}

\begin{verse}
Já nejsem svatej, ani ty nejsi svatá, \\
ale jablka z ráje bejvala jedovatá, \\
jenže hezky jsi hřála, když mi někdy bylo zima, \\
jó, dvě spálený srdce – Nagasaki, Hirošima.
\end{verse}

\begin{verse}
Tramvají dvojkou jezdíval jsem do Židenic, \\
z tak velký lásky většinou nezbyde nic, \\
z takový lásky jsou kruhy pod očima \\
a dvě spálený srdce – Nagasaki, Hirošima]
\end{verse}
  
\end{song}