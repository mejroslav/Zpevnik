\begin{song}{title = {Balada o očích topičových}}

Utichly továrny, utichly ulice, \\
usnuly hvězdy okolo měsíce \\
a z města celého v pozdní ty hodiny \\
nezavřel očí svých jenom dům jediný, \\
očí sých ohnivých, co do tmy křičí, \\
že za nimi uprostřed strojů, pák, kotlů a železných tyčí \\
dělníků deset své svaly železem propletlo, \\
aby se ruce a oči jim změnily ve světlo. \\

"Antoníne, topiči elektrárenský, \\
do kotle přilož!" \\

Antonín dnes, jak před lety dvaceti pěti, \\
železnou lopatou otvírá pec, \\ 
plameny rudé ztad syčí a letí, \\
ohnivá výheň a mládenec. \\
Antonín rukama, jež nad oheň ztuhly, \\
přikládá plnou lopatu uhlí, \\
a že jen z člověka světlo se rodí, \\
tak za uhlím vždycky kus očí svých hodí \\
a oči ty jasné a modré jak květiny \\
v praméncích drátů nad městem plují, \\
v kavárnách, v divadlech, nejraděj nad stolem rodiny \\
v radostná světla se rozsvěcují. \\

"Soudruzi, dělníci elektrárenští, \\
divnou ženu vám mám. \\
Když se jí do očí podívám, \\
pláče a říká, že člověk jsem prokletý, \\
že oči mám jiné, než jsem měl před lety. \\
Když prý šla se mnou k oltáři, \\
jako dva pecny velké a krásné byly, \\
teď prý jak v talířku prázdném mi na tváři \\
po nich jen drobinky dvě zbyly." \\

Smějí se soudruzi, Antonín s nimi \\
a uprostřed noci s hvězdami elektrickými \\
na svoje ženy si vzpomenou na chvíli, \\
které tak často si dětinsky myslily, \\
že muž na svět přišel, aby jim patřil. \\

A Antonín zas, jak před lety dvaceti pěti, \\
jen těžší lopatou otvírá pec. \\
Těžko je ženě vždy porozuměti, \\
má jinou pravdu a pravdivou přec. \\
Antonín očí květ v uhelné kusy \\
přikládá, neví snad o tom, - spíš musí, \\
neboť muž vždycky očima širokýma \\
se rozjet chce nad zemí a mít ji mezi nima \\
a jako slunce a měsíc z obou stran \\
paprsky lásky a úrody vjíždět do jejích bran. \\
V tu chvíli Antonín, topič mozolnatý, \\
poznal těch dvacet pět roků u pece, u lopaty, \\
v nichž oči mu krájel plamenný nůž, \\
a poznav, že stačí to muži, by zemřel jak muž, \\
zakřičel nesmírně nad nocí, nad světem vším: \\

"Soudruzi, dělníci elektrárenští, \\
slepý jsem, - nevidím!" \\

Sběhli se soudruzi \\
přestrašení celí, \\
dvěma nocemi \\
domů jej odváděli. \\
Na prahu jedné noci \\
žena s děckem sténá, \\
na prahu druhé noci \\
nebesa otevřená. \\

"Antoníne, \\
muži můj jediný, \\
proč tak se mi vracíš \\
v tyto hodiny? \\
Proč jsi se miloval \\
s tou holkou proklatou, \\
s milenkou železnou, \\
ohněm a lopatou? \\
Proč muž tu na světě \\
vždycky dvě lásky má, \\
proč jednu zabíjí \\
a na druhou umírá?" \\

Neslyší slepec, - do tmy se propadá \\
a tma jej objímá a tma jej opřádá, \\
raněné srdce už z hrudi mu odchází \\
hledat si ve světě jinačí obvazy, \\
však nad černou slepotou veselá lampa visí, \\
to není veselá lampa, - to jsou oči čísi, \\
to jsou oči tvoje, jež celému světu se daly, \\
aby tak nejjasněji viděly a nikdy neumíraly, \\
to jsi ty, topiči, vyrostlý nad těla zmučené střepy, \\
který se na sebe díváš, ač sám ležíš slepý. \\

Dělník je smrtelný, \\
práce je živá, \\
Antonín umírá, \\
žárovka zpívá: \\

Ženo má, - ženo má, \\
neplač!
\end{song}
\newpage