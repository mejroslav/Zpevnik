\begin{song}{
title = {Pražce}, 
interpret = {Pavel Dobeš}
}
\begin{verse}
Házím tornu na svý záda, feldflašku a sumky, \\
navštívím dnes kamaráda z železniční průmky.
\end{verse}

\begin{chorus}
Vždyť je jaro, zapni si kšandy, \\
pozdravuj vlaštovky a, muziko, ty hraj.
\end{chorus}
\begin{verse}
Vystupuji z vlaku, který mizí v dálce, \\
stojím v České Třebové a všude kolem pražce.
\end{verse}

\begin{verse}
Pohostil mě slivovicí, představil mě Mařce, \\
posadil mě na lavici z dubového pražce.
\end{verse}
\begin{verse}
Provedl mě domem - nikde kousek zdiva, \\
všude samej pražec, jen Máňa byla živá.
\end{verse}
\begin{chorus}
To je to jaro, zapni si kšandy, \\
pozdravuj vlaštovky a, muziko, ty hraj.
\end{chorus}
\begin{verse}
Plakáty nás informují:"Přijď pracovat k dráze, \\
pakliže ti vyhovují rychlost, šmír a saze."
\end{verse}
\begin{verse}
A jestliže jsi labužník a přes kapsu se praštíš, \\
upečeš i krávu na železničních pražcích.
\end{verse}
\begin{verse}
A naučíš se skákat tak, jak to umí vrabec, \\
když na nohu si pustíš železniční pražec.
\end{verse}
\begin{verse}
Když má děvče z Třebové rádo svého chlapce, \\
posílá mu na vojnu železniční pražce.
\end{verse}
\begin{verse}
A když děti zlobí, tak hned je doma mazec, \\
Děda Mráz jim nepřinese ani jeden pražec.
\end{verse}

\begin{verse}
A když máš dlouhý zkouškový a hlavu z toho ztráciš, \\
hlava může spočinout na železničních pražcích.
\end{verse}

\begin{verse}
Před děvčaty z Třebové chlubil jsem se silou, \\
pozvedl jsem pražec, načež odvezli mě s kýlou.
\end{verse}

\begin{verse}
Pamatuji pouze ještě operační sál, \\
pak praštili mě pražcem a já jsem tvrdě spal.
\end{verse} 
\begin{chorus}
A bylo jaro, zapni si kšandy, \\
lítaly vlaštovky a zelenal se háj.
\end{chorus}

\end{song}